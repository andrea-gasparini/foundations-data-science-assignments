\documentclass{article}
\usepackage[utf8]{inputenc}

\title{Foundations of Data Science \\ Image Filtering and Object Identification}
\author{Andrea Gasparini, Edoardo Di Paolo, Cirillo Atalla}
\date{October 2020}

\usepackage{natbib}
\usepackage{graphicx}
\usepackage{subcaption} % Per aggiungere caption ad immagini annidando in \begin[minipage]
\usepackage{cprotect}   % Per utilizzare \verb in caption delle immagini
\usepackage{hyperref}
\hypersetup{
    colorlinks=true,
    linkcolor=blue,
    filecolor=blue,      
    urlcolor=blue,
}
\usepackage{xcolor}

\begin{document}

\maketitle
{
  \hypersetup{linkcolor=black}
  \tableofcontents
}
\newpage

\section{Image Filtering}

\subsection{Question 1.d}
The effect of applying a filter can be studied by observing its \textit{impulse response}. Executing the following snippet we created a test image (\autoref{fig:test-image}) in which only the central pixel has a non-zero value:
\begin{verbatim}
  img_imp = np.zeros([27,27])
  img_imp[13, 13] = 1.0
  plt.figure(1), plt.imshow(img_imp, cmap='gray')
\end{verbatim}

\begin{figure}[ht]
    \centering
    \includegraphics[scale=0.4]{images/Q1.d-F1.png}
    \caption{Test Image}
    \label{fig:test-image}
\end{figure}

\noindent
Executing the following snippet we created 1D Gaussian and Gaussian derivative kernels, \verb|Gx| and \verb|Dx| respectively.
\begin{verbatim}
  sigma = 7.0
  [Gx, x] = gauss_module.gauss(sigma)
  [Dx, x] = gauss_module.gaussdx(sigma)
\end{verbatim}
We applied the following filter combinations:
\begin{enumerate}
    \item First \verb|Gx|, then $ \verb|Gx|^T $
    \item First \verb|Gx|, then $ \verb|Dx|^T $
    \item First $ \verb|Dx|^T $, then \verb|Gx|
    \item First \verb|Dx|, then $ \verb|Dx|^T $
    \item First \verb|Dx|, then $ \verb|Gx|^T $
    \item First $ \verb|Gx|^T $, then \verb|Dx|
\end{enumerate}

\begin{figure}[ht]
    \centering
    \includegraphics[width=\textwidth]{images/Q1.d-F2.png}
    \caption{Applying filter combinations}
    \label{fig:filter-combination}
\end{figure}

\noindent
As we can see in \autoref{fig:filter-combination}, the first filter combination [...]
\newline
{\color{red} \large \textbf{TODO}} \textit{What happens when you apply the following filter combinations?}

\subsection{Question 1.e}
We implemented a \verb|gaussderiv| method that takes an input image and generates two copies of it, smoothed according to a standard deviation $\sigma$ and derived in the directions $x$ and $y$ respectively.
\newline
\newline
The results of applying \verb|gaussderiv|, with $\sigma = 7.0$, to the provided example images (\verb|graf.png| and \verb|gantrycrane.png|) are shown in Figures \ref{fig:gaussderiv-graf.png} and \ref{fig:gaussderiv-gantrycrane.png}.

\begin{figure}[ht]
    \centering
    \begin{minipage}{.5\textwidth}
      \centering
      \includegraphics[width=.6\linewidth]{images/Q1.e-graf.png}
      \captionof{figure}{graf.png}
      \label{fig:graf.png}
    \end{minipage}%
    \begin{minipage}{.5\textwidth}
      \centering
      \includegraphics[width=.6\linewidth]{images/Q1.e-gantrycrane.png}
      \captionof{figure}{gantrycrane.png}
      \label{fig:gantrycrane.png}
    \end{minipage}
\end{figure}

\begin{figure}[ht]
    \centering
    \includegraphics[width=\textwidth]{images/Q1.e-graf-gaussderived.png}
    \cprotect\caption{Results of applying \verb|gaussderiv| on \verb|graf.png|}
    \label{fig:gaussderiv-graf.png}
\end{figure}

\begin{figure}[ht]
    \centering
    \includegraphics[width=\textwidth]{images/Q1.e-gantrycrane-gaussderived.png}
    \cprotect\caption{Results of applying \verb|gaussderiv| on \verb|gantrycrane.png|}
    \label{fig:gaussderiv-gantrycrane.png}
\end{figure}

\noindent
{\color{red} \large \textbf{TODO}} \textit{Comment on the output in your report}.
\newline
{\color{red} \large \textbf{TODO}} \textit{Consider also why smoothing an image is important before applying the derivative filter}.

\end{document}
