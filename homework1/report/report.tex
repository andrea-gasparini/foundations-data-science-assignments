\documentclass{article}
\usepackage[utf8]{inputenc}

\title{Foundations of Data Science \\ Image Filtering and Object Identification}
\author{Andrea Gasparini, Edoardo Di Paolo, Cirillo Atalla}
\date{October 2020}

\usepackage{natbib}
\usepackage{graphicx}
\usepackage{subcaption} % Per aggiungere caption ad immagini annidando in \begin[minipage]
\usepackage{cprotect}   % Per utilizzare \verb in caption delle immagini
\usepackage{hyperref}
\usepackage{longtable}

\hypersetup{
    colorlinks=true,
    linkcolor=blue,
    filecolor=blue,      
    urlcolor=blue,
}
\usepackage{xcolor}

\begin{document}

\maketitle
{
  \hypersetup{linkcolor=black}
  \tableofcontents
}
\newpage

\section{Image Filtering}

\subsection{Question 1.d}
The effect of applying a filter can be studied by observing its \textit{impulse response}. Executing the following snippet we created a test image (\autoref{fig:test-image}) in which only the central pixel has a non-zero value:
\begin{verbatim}
  img_imp = np.zeros([27,27])
  img_imp[13, 13] = 1.0
  plt.figure(1), plt.imshow(img_imp, cmap='gray')
\end{verbatim}

\begin{figure}[ht]
    \centering
    \includegraphics[scale=0.4]{images/Q1.d-F1.png}
    \caption{Test image}
    \label{fig:test-image}
\end{figure}

\noindent
Executing the following snippet we created 1D Gaussian and Gaussian derivative kernels, \verb|Gx| and \verb|Dx| respectively.
\begin{verbatim}
  sigma = 7.0
  [Gx, x] = gauss_module.gauss(sigma)
  [Dx, x] = gauss_module.gaussdx(sigma)
\end{verbatim}
We applied the following filter combinations:
\begin{enumerate}
    \item First \verb|Gx|, then $ \verb|Gx|^T $
    \item First \verb|Gx|, then $ \verb|Dx|^T $
    \item First $ \verb|Dx|^T $, then \verb|Gx|
    \item First \verb|Dx|, then $ \verb|Dx|^T $
    \item First \verb|Dx|, then $ \verb|Gx|^T $
    \item First $ \verb|Gx|^T $, then \verb|Dx|
\end{enumerate}

\begin{figure}[ht]
    \centering
    \includegraphics[width=\textwidth]{images/Q1.d-F2.png}
    \caption{Applying filter combinations}
    \label{fig:filter-combination}
\end{figure}

\noindent
As we can see in \autoref{fig:filter-combination}, the first filter combination is the result of the gaussian filter applied first on the rows and then on the columns. So we compute two 1D convolution instead of one 2D convolution due to the separability of gaussian filter.
\newline
The second and third filter combinations are the same, there is no difference in applying \verb|Gx| and then $\verb|Dx|^T$ or viceversa. We find an edge when we apply the first derivative filter.
\newline
In the fourth filter combination there are some edges. We can see the changes from white to black and viceversa.
\newline
The fifth and sixth filter combinations are the same. This is the same case of 2nd and 3rd, but with inverted axis; there is no difference in applying \verb|Dx| and then $\verb|Gx|^T$ or viceversa.


\subsection{Question 1.e}
We implemented a \verb|gaussderiv| method that takes an input image and generates three copies of it. The first two are smoothed according to a standard deviation $\sigma$ and derived in the directions $x$ and $y$ respectively; the third is a combination of them, obtained as $f(img_x, img_y) = \sqrt{img_x^2 + img_y^2}$ where $img_x$ and $img_y$ are corresponding pixels of the two filtered images.

\noindent
\newline
The results of applying \verb|gaussderiv|, with $\sigma = 7.0$, to the provided example images (\verb|graf.png| and \verb|gantrycrane.png|) are shown in Figures \ref{fig:gaussderiv-graf.png} and \ref{fig:gaussderiv-gantrycrane.png}.

\begin{figure}[ht]
    \centering
    \begin{minipage}{.5\textwidth}
      \centering
      \includegraphics[width=.6\linewidth]{images/Q1.e-graf.png}
      \cprotect\caption{\verb|graf.png|}
      \label{fig:graf.png}
    \end{minipage}%
    \begin{minipage}{.5\textwidth}
      \centering
      \includegraphics[width=.6\linewidth]{images/Q1.e-gantrycrane.png}
      \cprotect\caption{\verb|gantrycrane.png|}
      \label{fig:gantrycrane.png}
    \end{minipage}
\end{figure}

\begin{figure}[ht]
    \centering
    \includegraphics[width=\textwidth]{images/Q1.e-graf-gaussderived.png}
    \cprotect\caption{Results of applying \verb|gaussderiv| on \verb|graf.png|}
    \label{fig:gaussderiv-graf.png}
    
    \includegraphics[width=\textwidth]{images/Q1.e-gantrycrane-gaussderived.png}
    \cprotect\caption{Results of applying \verb|gaussderiv| on \verb|gantrycrane.png|}
    \label{fig:gaussderiv-gantrycrane.png}
\end{figure}

\noindent
Smoothing an image is important because with this process we can leave out the noise of the images.
\newline
\newline
From the left plot in Figures \ref{fig:gaussderiv-graf.png} and \ref{fig:gaussderiv-gantrycrane.png} we can see the edges on the vertical axis, while in the second one we can see the edges on the horizontal axis. The third one is the combination of the previous two.

\newpage
\section{Object Identification}

\subsection{Question 3.c}
For the tests we used different combinations of distance type ($\chi^2$, \textit{intersect} and \textit{l2}), histogram type (\textit{grayvalue, rgb, rg} and \textit{dxdy}) and number of bins.
\newline
\newline
These are the results we got:

\begin{longtable}{|l|l|l|l|l|}
    \hline
    \textbf{Distance type} & \textbf{Histogram type} & \textbf{Num bins} & \textbf{Correct} & \textbf{Rate}     \\ \hline
    intersect     & rgb            & 25       & 81      & 0.910112 \\ \hline
    intersect     & rgb            & 15       & 78      & 0.876404 \\ \hline
    intersect     & rg             & 25       & 75      & 0.842697 \\ \hline
    intersect     & rg             & 15       & 74      & 0.831461 \\ \hline
    intersect     & rgb            & 30       & 72      & 0.808989 \\ \hline
    intersect     & rgb            & 20       & 71      & 0.797753 \\ \hline
    intersect     & rgb            & 10       & 70      & 0.786517 \\ \hline
    intersect     & rg             & 20       & 65      & 0.730337 \\ \hline
    intersect     & rg             & 30       & 65      & 0.730337 \\ \hline
    intersect     & rg             & 10       & 62      & 0.696629 \\ \hline
    chi2          & rgb            & 5        & 60      & 0.674157 \\ \hline
    chi2          & rgb            & 10       & 59      & 0.662921 \\ \hline
    chi2          & rg             & 5        & 57      & 0.640449 \\ \hline
    chi2          & rgb            & 15       & 55      & 0.617978 \\ \hline
    l2            & rgb            & 10       & 54      & 0.606742 \\ \hline
    chi2          & rg             & 15       & 53      & 0.595506 \\ \hline
    l2            & rg             & 16       & 53      & 0.595506 \\ \hline
    chi2          & rg             & 10       & 52      & 0.584270 \\ \hline
    l2            & rg             & 10       & 52      & 0.584270 \\ \hline
    l2            & rg             & 15       & 52      & 0.584270 \\ \hline
    chi2          & rg             & 25       & 50      & 0.561798 \\ \hline
    chi2          & rg             & 20       & 48      & 0.539326 \\ \hline
    chi2          & rgb            & 20       & 46      & 0.516854 \\ \hline
    intersect     & dxdy           & 70       & 46      & 0.516854 \\ \hline
    intersect     & grayvalue      & 20       & 46      & 0.516854 \\ \hline
    intersect     & grayvalue      & 35       & 46      & 0.516854 \\ \hline
    chi2          & rgb            & 25       & 45      & 0.505618 \\ \hline
    intersect     & grayvalue      & 10       & 45      & 0.505618 \\ \hline
    intersect     & grayvalue      & 30       & 45      & 0.505618 \\ \hline
    intersect     & grayvalue      & 25       & 43      & 0.483146 \\ \hline
    intersect     & grayvalue      & 70       & 43      & 0.483146 \\ \hline
    l2            & rgb            & 15       & 42      & 0.471910 \\ \hline
    chi2          & rg             & 30       & 41      & 0.460674 \\ \hline
    chi2          & dxdy           & 150      & 41      & 0.460674 \\ \hline
    chi2          & grayvalue      & 10       & 41      & 0.460674 \\ \hline
    intersect     & dxdy           & 35       & 40      & 0.449438 \\ \hline
    intersect     & grayvalue      & 15       & 40      & 0.449438 \\ \hline
    chi2          & dxdy           & 70       & 39      & 0.438202 \\ \hline
    l2            & dxdy           & 70       & 39      & 0.438202 \\ \hline
    l2            & grayvalue      & 10       & 39      & 0.438202 \\ \hline
    intersect     & dxdy           & 25       & 37      & 0.415730 \\ \hline
    intersect     & dxdy           & 30       & 37      & 0.415730 \\ \hline
    chi2          & rgb            & 30       & 36      & 0.404494 \\ \hline
    chi2          & grayvalue      & 15       & 36      & 0.404494 \\ \hline
    chi2          & grayvalue      & 25       & 36      & 0.404494 \\ \hline
    chi2          & grayvalue      & 30       & 36      & 0.404494 \\ \hline
    l2            & grayvalue      & 15       & 36      & 0.404494 \\ \hline
    l2            & grayvalue      & 30       & 36      & 0.404494 \\ \hline
    intersect     & dxdy           & 20       & 35      & 0.393258 \\ \hline
    chi2          & grayvalue      & 5        & 35      & 0.393258 \\ \hline
    chi2          & grayvalue      & 20       & 35      & 0.393258 \\ \hline
    chi2          & dxdy           & 30       & 34      & 0.382022 \\ \hline
    chi2          & dxdy           & 25       & 33      & 0.370787 \\ \hline
    l2            & dxdy           & 30       & 33      & 0.370787 \\ \hline
    chi2          & dxdy           & 20       & 32      & 0.359551 \\ \hline
    intersect     & dxdy           & 15       & 31      & 0.348315 \\ \hline
    chi2          & grayvalue      & 70       & 30      & 0.337079 \\ \hline
    chi2          & dxdy           & 15       & 29      & 0.325483 \\ \hline
    l2            & dxdy           & 15       & 29      & 0.325483 \\ \hline
    l2            & grayvalue      & 70       & 28      & 0.314607 \\ \hline
    chi2          & grayvalue      & 150      & 27      & 0.303371 \\ \hline
    chi2          & dxdy           & 5        & 25      & 0.280899 \\ \hline
    chi2          & dxdy           & 10       & 25      & 0.280899 \\ \hline
    l2            & dxdy           & 10       & 25      & 0.280899 \\ \hline
    intersect     & dxdy           & 10       & 23      & 0.258427 \\ \hline 
\end{longtable}

\noindent
As we can see that the best measure is the intersect with \textit{rgb} and \textit{rg} histogram types but with the other types (\textit{dxdy, grayvalue)} it is not very good.
We can see also that the l2 distance is not good to recognize objects.
\newline
\newline
About the histograms the best ones are rg and rgb due to their robustness. The color histograms are really useful with the intersect distance while with the other distance types the recognition rate is similar to the dxdy and grayvalue histogram types.
We tried with different number of bins in a range from 5 to 150.

\end{document}
